\documentclass{article}
\usepackage[utf8]{inputenc}
\usepackage{parskip}
\usepackage{hyperref}

\author{Emil Njor}
\title{tinyML Energy Harvesting Systems}
\date{\today}

\begin{document}
\maketitle
\textbf{Supervisor:} Xenofon Fafoutis \href{mailto:xefa@dtu.dk}{\textless{}xefa@dtu.dk\textgreater}

\textbf{Co-Supervisor:} Emil Njor \href{mailto:emjn@dtu.dk}{\textless{}emjn@dtu.dk\textgreater}

\textbf{Background:} tinyML is a promising research area concerned with running machine learning models on ultra-low power devices, typically in the range of milliwatts or below.
The state of the art of tinyML involves training machine learning models on larger computers (e.g., a laptop computer), and subsequently applying optimizations and deploying the models on ultra-low power devices.
tinyML has interesting use cases in rural areas where a working network connection cannot always be guaranteed or where a network is undesirable due to latency or privacy concerns.

Energy Harvesting Systems are systems that harvest energy from their environment.
A simple example of this is a system that is powered using an attached solar panel.
More complex and interesting systems also exist where e.g., the push of a button powers a system that responds to the push.

\textbf{Project Description:} The promise of tinyML is often that ultra-low power devices can be powered by a small battery for an extended period of time.
But what if we do not even need a battery to power the device?
Ultra-low power devices by definition consume very low power, which should enable them to be powered by the energy in its surrounding environment. Solar-powered tinyML has been thought of before, but some more interesting and unique energy harvesting systems could be constructed. E.g., vibration powered predictive maintenance.

\textbf{Recommended Background Knowledge:} Embedded Systems, Machine Learning, C++ \& Python.

\end{document}